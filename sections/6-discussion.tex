\chapter{Discussion}
Tractogram filtering methods like COMMIT exploit the diffusion data and improve the plausibility of 
the streamlines by removing the redundant and wrong streamlines in a way. 
However, COMMIT method is still limited. Delighted by the study on SIFT \cite{hainAssessingStreamlinePlausibility2022}, 
we purpose the rCOMMIT experiment to 
investigate the biases and limitations of COMMIT. Meanwhile, this experiment provides a way 
to extract more plausible streamlines based on filtering methods. 
In addition, the comparison study between COMMIT and SIFT is conducted. 
And neural networks are used to further investigate the characteristics of the streamlines from 
different groups, which show the potential of deep learning in filtering field. 
So the following sections will summarize, compare and explain the results. 
The possible errors and limitations during 
the study are also discussed. Moreover, the future work of this project are discussed.

\section{rCOMMIT}
From the results in last section, COMMIT has shown that the same streamline can be accepted or rejected in different runs, which is 
how the implausible group is defined in the study. This phenomenon indicates that the way COMMIT realizes filtering doesn't depend
on the true anatomy of the streamlines. And the process of filtering tries to find a subset of streamlines that can improve 
the consistency between tractogram and diffusion data with global optimization approach \cite{hainAssessingStreamlinePlausibility2022}.
Instead of assessing an individual streamline, COMMIT tries to fit with all streamlines as input. 
As shown in the results, with smaller input tractogram, the proportion of accepted streamlines gets higher 
(7.0\% for the whole tractogram vs. 28.5\% for the smallest subset). From the whole tractogram to the smallest subset, the proportion of 
the rejected streamlines changes from  93.4\% to 27.2\%. 
A possible explanation is that the smaller subsets might contain fewer redundant streamlines, so that the accepted proportion increases
due to the lower base number. 

Also, the ground truth of the anatomy of the streamlines is deficient for most of the tractograms, 
which brings another limitation to the filtering method. For these limitations, rCOMMIT method is designed to give an independent
assessment on each streamline. The independence of the assessments comes from the randomized sampling and the separation between runs.  
Although the ground truth is still lacking for rCOMMIT, the classified streamlines from rCOMMIT are statistically plausible which avoid 
the bias of size compared with COMMIT. 

\section{Comparison Between COMMIT and SIFT}

From the results of rSIFT, we conclude that SIFT method in general is more strict in filtering than COMMIT. From the results, SIFT 
is prone to reject more streamlines when applying both methods on the same tractogram.  
The difference from the filtering theories may explain this. 
Both methods seek for a set of streamlines that could best describe the data. For COMMIT, the data to be reconstructed by the streamlines is measured diffusion data, while 
SIFT aims to associate the streamline densities with FOD lobe integrals, which is a representative of the diffusion data. Besides,
during filtering, SIFT removes streamlines iteratively to better fit the measured data, so that this process is irreversible. And it's 
also possible that the solution reaches local minima instead of a global solution.


From the venn diagrams, there is a big intersection between the implausible groups from rSIFT and rCOMMIT, 
and most of the implausible streamlines from rCOMMIT are also included in by rSIFT. For the plausible 
group, the rSIFT has a smaller group of streamlines. The proportion of the intersection to plausible groups is not 
as big as the proportion of the implausible intersection. It indicates that although the theories of filtering differ from each other,
the results of implausible streamlines from both methods are largely overlapping. 
The theories of methods may also explain this. The iterative and irreversible removal of streamlines in SIFT filters out more streamlines than COMMIT,
and COMMIT computes the weight for each streamline which represents their contribution to the reconstruction of the diffusion signal. 
The less plausible streamlines may of lower weight, but they can still be in the reconstruction, but these streamlines might be removed by SIFT.


\section{Classification}

\section{Source of Error}
For the higher acceptance in smaller subsets, the explanation indicates that the proportion of redundant streamlines might influence
the acceptance. 
However, the random sampling step 


\section{Future Work}








