\chapter{Discussion}
Tractogram filtering methods like COMMIT exploit the diffusion data and improve the plausibility of 
the streamlines by removing the redundant and wrong streamlines in a way. 
However, COMMIT method is still limited. Delighted by the study on SIFT \cite{hainAssessingStreamlinePlausibility2022}, 
we purpose the rCOMMIT experiment to 
investigate the biases and limitations of COMMIT. Meanwhile, this experiment provides a way 
to extract more plausible streamlines based on filtering methods. 
In addition, the comparison study between COMMIT and SIFT is conducted. 
And neural networks are used to further investigate the characteristics of the streamlines from 
different groups, which show the potential of deep learning in filtering field. 
So the following sections will summarize, compare and explain the results. 
The possible errors and limitations during 
the study are also discussed. Moreover, the future work of this project are discussed.

\section{rCOMMIT}
From the results in results section, COMMIT has shown that the same streamline can be accepted or rejected in different runs, which is 
how the implausible group is defined in the study. This phenomenon indicates that the way COMMIT realizes filtering doesn't depend
on the anatomy of each streamline. And the process of filtering tries to find a subset of streamlines that can improve 
the consistency between tractogram and diffusion data with global optimization approach \cite{hainAssessingStreamlinePlausibility2022}.
Instead of assessing an individual streamline, COMMIT tries to fit with all streamlines as input. 
As shown in the results, with smaller input tractogram, the proportion of accepted streamlines gets higher 
(7.0\% for the whole tractogram vs. 28.5\% for the smallest subset). From the whole tractogram to the smallest subset, the proportion of 
the rejected streamlines decreases from  93.4\% to 27.2\%. 
A possible explanation is that the smaller subsets might contain fewer redundant streamlines, so that the accepted proportion increases
due to the lower base number. 

Also, the ground truth of the anatomy of the streamlines is deficient for most of the tractograms, 
which brings another limitation to the filtering method. For these limitations, rCOMMIT method is designed to give an independent
assessment on each streamline. The independence of the assessments comes from the randomized sampling and the separation between runs.  
Although the ground truth is still lacking for rCOMMIT, the classified streamlines from rCOMMIT are statistically plausible which avoid 
the bias of size compared with COMMIT. 

Besides, the bias on length during the filtering by COMMIT is also confirmed in this study.
From the distribution of the number of streamlines with the respect to the length, it's obvious that 
there is greater variety in the rejected streamlines by rCOMMIT than the accepted group. 
This phenomenon is also found in SIFT that it tends to filter out longer streamlines.
A guess on this problem is that the longer streamlines are prone to give more error during the optimization process
to find the best set of streamlines. To reach the minima, these streamlines are more likely to be filtered. 
However, from the current results, there isn't a very convincing explanation for it. 

\section{Comparison Between rCOMMIT and rSIFT}

Comparing the results from rSIFT with rCOMMIT, we conclude that SIFT method in general is more strict in filtering than COMMIT. From the results, SIFT 
is prone to reject more streamlines when applying both methods on the same tractogram.  
The difference from the filtering theories may explain this. 
Both methods seek for a set of streamlines that could best describe the data. For COMMIT, the data to be reconstructed by the streamlines is measured diffusion data, while 
SIFT aims to associate the streamline densities with FOD lobe integrals, which is a representative of the diffusion data. Besides,
during filtering, SIFT removes streamlines iteratively to better fit the measured data, so that this process is irreversible. And it's 
also possible that the solution reaches local minima instead of a global solution.


From the venn diagrams, there is a big intersection between the implausible groups from rSIFT and rCOMMIT, 
and most of the implausible streamlines from rCOMMIT are also included in by rSIFT. For the plausible 
group, the rSIFT has a smaller group of streamlines. The proportion of the intersection to plausible groups is not 
as big as the proportion of the implausible intersection. Although the theories of filtering differ from each other,
the results of implausible streamlines from both methods are largely overlapping. 
The theories behind these methods may also explain it. The iterative and irreversible removal of streamlines in SIFT filters out more streamlines than COMMIT,
and COMMIT computes the weight for each streamline which represents their contribution to the reconstruction of the diffusion signal. 
The less plausible streamlines may of lower weights, but they can still be in the reconstruction, but these streamlines might be removed by SIFT.


\section{Classification}

Neural networks are exploited for different classification tasks and to show the potential of deep learning methods in filtering.
The binary classifiers trained on the plausible and implausible groups from rCOMMIT obtained an average accuracy of 78\%, which indicates the 
structural differences between these two groups that can be learned by the neural networks. Since the whole process of rCOMMIT is very time-comsuming,
the binary classifiers applied on the unknown subject also shows its ability in distinguishing plausible streamlines from the implausible ones.

The binary classifiers trained on the plausible and implausible intersections between rCOMMIT and rSIFT show the even higher accuracy than
the one trained on the pseudo ground truth from rCOMMIT. As mentioned, compared with their own groups, the proportion of the plausible intersection is not as big as the 
implausible intersection. Because the performance of the classifiers is improved from the new training data, we conclude that within the same
structure of the neural networks and hyperparameters there is greater difference between plausible and implausible intersections.
Since the implausible intersection is very closed to the implausible group from rCOMMIT, the difference mainly comes from the plausible intersection.
Based on this combined information from two methods, we conclude that the overlapping streamlines are likely to survive from filtering and be more plausible.

% (to do) When the classifiers based on the intersection data are applied on the unknown subject, ...  

The binary classifiers are also respectively trained on the plausible non-intersection groups and implausible non-intersection groups from both methods.
We assume that the performance of classifiers can indirectly indicate the difference between the non-intersection parts.
With the same hyperparameters and the structure, 
the mean accuracy of the classifiers trained on the plausible non-intersection groups is higher than the other.
It indicates that it's easier for the same model to learn the difference between the plausible non-intersection groups,
so there might be a greater difference between the plausible streamlines that are chosen by two methods.
However, from the venn diagrams show the difference in the amounts of streamlines in different groups which might
also explain the performance of classifiers. 
Because the implausible non-intersection group from rCOMMIT is rather small compared to the one from rSIFT.
And the proportion between the two plausible non-intersection groups is relatively balanced.
Although during the training the data generator is used to balance the two input classes, there still
might be not enough various streamlines for the model to study. 
In the end, we can't come to any conclusion about the difference in non-intersection groups.


\section{Source of Error}

In the design of rCOMMIT, some drawbacks should be emphasized. The concept of AR helps classify the streamlines, but the 
computation of the AR depends on the base number of runs by COMMIT for each streamline. 
Although in each size level of subsets the streamlines are on average assessed by COMMIT 5 times, some streamlines still get fewer assessments,
which might cause bias to the AR value. One possible solution is to choose the streamlines with 5 times assessments before
generating the pseudo ground truth. It ensures that the ARs are computed with the same denominator, 
but the total number of streamlines from the pseudo ground might differ between subjects.
The second solution is increasing the sizes of subsets in the experiments, which will reduce the impact from the bias but not remove it.
To assess this bias, a distribution of classification results from streamlines with 5 votes can help.




\section{Future Work}

There are several questions remained after this study. The bias on the length of the filtering methods
is found but lack of explanations. Deep learning models can be used to investigate this question indirectly as well.
For example, the accuracy of the classifiers trained on streamlines of different lengths might give some evidences.  
Besides, the plausible groups from both two methods can be an interesting angle to 
study the difference between filtering methods.  

This classifier built in this study shows the potential of DL methods, so it's promising to have applications 
based on DL for filtering in the future. Also, there are some existing ground truths for some tractograms that 
can be used for training. The time and computation of the filtering process can be improved by DL.   

In addition, there is plenty of work can be addressed in improving the accuracy of the networks.
For example, using more information about the structure of the streamlines to the network
and trying different types of neural network are worthy to be studied. Since only the 1-D information of the 
coordinates of sampling points are used in this study, more structural features from the input data might 
help the network learn more about the anatomy of plausible streamlines.

Besides COMMIT and SIFT, there are other filtering methods such as LiFE, SIFT2, COMMIT2 to be investigated.
The difference and similarity between the methods are not very clear. Similar experiments can be conducted and 
more information about filtering methods can help achieve better quality of tractograms for further research.






