\chapter{Conclusion}
In this study, we proposed rCOMMIT as a method to study the performance of 
filtering method COMMIT and give assessment on individual streamlines. 
The results from the rCOMMIT shows the biases during the filtering and comfirm the 
biased assessment on individual streamlines. 

Additionally, the pseudo ground truth for the input data is generated with rCOMMIT,
which divides the input streamlines into three groups (plausible, implausible, and inconclusive).
Based on this data, deep learning-based classifiers are trained to distinguish different streamlines.
The classifiers are applied to both binary and multi-classes classification tasks and 
show the potential of deep learning method in filtering. 
It learns the characteristics of different type of streamlines 
and speeds up the filtering process.  

Moreover, with findings from the previous research on SIFT, comparisons between two filtering
methods are conducted. It's found that the bias on length existing in both methods.
And the SIFT method is stricter than COMMIT in filtering. The intersection data from both methods also
provides plausible information on the streamlines. And the deep learning methods are used to 
indirectly investigate the difference between groups. 

