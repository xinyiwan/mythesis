\chapter{Theoretical Background}


In this chapter, theoretical background of this project will be introduced in details. To better introduce the importance of the filtering, a big picture of the pipeline of tractography will be first presented. 

\section{Diffusion MRI}

Diffusion MRI is a non-invasive method to track the random microscopic motion (or diffusion) of water molecules in biological tissues. 
As molecules interact with different obstacles as they diffuse throughout tissues, 
dMRI provides insight into the microscopic details of tissue architecture \cite{newmanChapterMorphologicalBrain2014}. 
The water molecules moving in the brain tissues may cross or interact with many tissue components such as cell membranes, fibres and macromolecules. \cite{lebihanLookingFunctionalArchitecture2003}
It means the movement of these water molecules is impeded by the tissues. 
Compared to the water molecules that can move freely, 
the diffusion distance of constraint molecules is reduced and disobeys the standard 3-dimentinal Gaussian. 
This phenomenon is described in more details in the \ref{sec:dmri} as well.

\section{Fibre Orientation}
By measuring the diffusion signal, the orientations of fibres can be interpreted which provide the 'density' information of the tissue of interest.
Then in the next step, the local orientation information can be pieced together to infer the long-range pathways connecting distant regions of the brain \cite{lebihanLookingFunctionalArchitecture2003}, 
which is realized with tractography. 
Currently, diffusion tensor imaging (DTI) is still the most widely used method for assessing WM orientation and organization \cite{basserEstimationEffectiveSelfDiffusion1994}.
However, when meeting more complex structures such multiple fibres crossing or passing each other, the tractography based on DTI is prone to 
generate implausible results. Therefore, to achieve better tractograms from tracking algorithms, it's essential to have diffusion model that can 
well represent the orientations. Various methods have been purposed to assess the orientations from diffusion signals, which are introduced in \ref{sec:dmri}.

In this section, the high angular resolution diffusion imaging (HARDI) protocol and the method related to it are explained. 
HARDI measures a great number of orientations from diffusion signals.


\section{Tractography}

Test \autocite{dhollanderFixelbasedAnalysisDiffusion2021}


\section{Fibre Filtering Methods}

\subsection{COMMIT}

\subsubsection{Filtering Theory}

\subsubsection{Existing Problems}

\subsection{SIFT}

\subsubsection{Filtering Theory}

\subsubsection{Filtering Theory}

\section{Deep Learning Based Classification}

\subsection{Multiple Layer P}
\subsection{RNN}




