\newpage
\thispagestyle{plain}
~\\
\vfill
{ \setstretch{1.1}
	\subsection*{Authors}
	Xinyi Wan <xinyiwan@kth.se>\\
	School of Engineering Sciences in Chemistry, Biotechnology and Health\\
	KTH Royal Institute of Technology
	
	\subsection*{Place for Project}
	Division of Biomedical Imaging\\
	KTH Royal Institute of Technology
	
	\subsection*{Examiner}
	Chunliang Wang\\
	HÄLSOVÄGEN 11 C, HUDDINGE \\
	KTH Royal Institute of Technology
	
	\subsection*{Supervisor}
	Rodrigo Moreno\\
	HÄLSOVÄGEN 11 C, HUDDINGE\\
	KTH Royal Institute of Technology
	~
}


\newpage
\thispagestyle{plain}
%%%%%%%%%%%%%%%%%%%%%%%%%%%%%%%%%%%%
%%  The English abstract          %%
%%%%%%%%%%%%%%%%%%%%%%%%%%%%%%%%%%%%
\chapter*{Abstract}
%%%%%%%%%%%%%%%%%%%%%%%%%%%%%%%%%%%%

Tractography is a tool and widely used in the brain connectivity study from
diffusion magnetic resonance imaging data. However, lack of ground truth and plenty of 
anatomically implausible streamlines in the tractograms have caused challenges and concerns.
Tractogram filtering methods have been developed to remove the faulty connections.
In this study, we focus on one of these filtering methods, \textit{Convex Optimization Modeling for Microstructure Informed Tractography} (COMMIT),
which tries to find a set of streamlines that best reconstruct the diffusion magnetic resonance imaging data with global optimization approach.
There are biases with this method when assessing individual streamlines. 
So a method named randomized COMMIT(rCOMMIT) is proposed to obtain multiple assessments for each streamline.
The acceptance rate from this method is introduced to the streamlines and divides them into three groups, which 
are regarded as pseudo ground truth from rCOMMIT. Therefore, the neural networks are able to train on the pseudo ground truth on 
classification tasks. The trained classifiers distinguish the obtained groups of plausible and implausible streamlines with accuracy around 77\%.
Following the same methodology, the results from rCOMMIT and randomized SIFT are compared. 
The intersections between two methods are analyzed with neural networks as well, which achieve accuracy around 87\% in binary task between plausible and implausible streamlines.



\subsection*{Keywords}
Tractography, dMRI, Filtering Methods, Deep Learning, Classification




\newpage
\thispagestyle{plain}
%%%%%%%%%%%%%%%%%%%%%%%%%%%%%%%%%%%%
%%	 The Swedish abstract         %%
%%%%%%%%%%%%%%%%%%%%%%%%%%%%%%%%%%%%
\chapter*{Abstract}
%%%%%%%%%%%%%%%%%%%%%%%%%%%%%%%%%%%%
Traktografi är ett verktyg och används ofta i hjärnanslutningsstudien från
magnetisk resonansavbildningsdata för diffusion. Dock brist på grundsanning och gott om
anatomiskt osannolika strömlinjer i traktogrammen har orsakat utmaningar och bekymmer.
Traktogramfiltreringsmetoder har utvecklats för att ta bort de felaktiga anslutningarna.
I denna studie fokuserar vi på en av dessa filtreringsmetoder, \textit{Convex Optimization Modeling for Microstructure Informed Tractography} (COMMIT),
som försöker hitta en uppsättning strömlinjer som bäst rekonstruerar diffusionsmagnetisk resonansavbildningsdata med global optimeringsmetod.
Det finns fördomar med denna metod när man bedömer individuella effektiviseringar.
Så en metod som heter randomized COMMIT(rCOMMIT) föreslås för att få flera bedömningar för varje effektivisering.
Acceptansgraden från denna metod introduceras till effektiviseringarna och delar upp dem i tre grupper, som
betraktas som pseudogrund sanning från rCOMMIT. Därför kan de neurala nätverken träna på pseudogrunden sanning på
klassificeringsuppgifter. De tränade klassificerarna särskiljer de erhållna grupperna av rimliga och osannolika strömlinjer med en noggrannhet runt 77\%.
Enligt samma metodik jämförs resultaten från rCOMMIT och randomiserat SIFT.
Skärningspunkterna mellan två metoder analyseras också med neurala nätverk, som uppnår en noggrannhet runt 87\% i binär uppgift mellan rimliga och osannolika strömlinjer.
\subsection*{Nyckelord}
Traktografi, dMRI, filtreringsmetoder, djupinlärning, klassificering

\newpage
\thispagestyle{plain}
\chapter*{Acknowledgements}
I would like to express my sincere gratitude to my supervisor Rodrigo Moreno for his invaluable guidance and support 
throughout my master study. Without his constant encouragement and insightful feedback I would not have been able to complete this project.
Moreover, I would like to thank the PhD students in our research group at KTH, 
Mehdi Astaraki, Fabian Sinzinger, Simone Bendazzoli, Jingru Fu for providing me feedbacks and insights when I struggled with my project.
Many thanks should also go to the master students in our group, Karine Louis, Lasse Stahnke and Yuqi Zheng. 
Our discussions and fika time were not only valuable but also enjoyable.
Finally, I want to express my deepest gratitude to my family for their support and encouragement throughout my academic journey.

\newpage

% \chapter*{Acronyms}


% \begin{acronym}
% \acro{dMRI}{Diffusion Magnetic Resonance Imaging}
% \acro{USA}{United States of America}

% \end{acronym}

\begin{acronym}
    \acro  {iot}   [IoT]   {Internet-of-Things}
    \acro  {icann} [ICANN] {Internet Corporation for Assigned names and Numbers}
    \acro  {iso}   [ISO]   {International Organization for Standardization}
    \acro  {ietf}  [IETF]  {Internet Engineering Task Force}    % will not be listed, as it is no
\end{acronym}


\newpage

\etocdepthtag.toc{mtchapter}
\etocsettagdepth{mtchapter}{subsection}
\etocsettagdepth{mtappendix}{none}
\thispagestyle{plain}
\tableofcontents

\newpage


